\chapter{Introdução}
\label{ch:intro}

Este projeto tem como objetivo implementar uma API de
algoritmos relacionados a Grafos na linguagem de programação Rust. A
especificação da API segue as funcionalidades descritas na
definição da Avaliação 01 da disciplina de Grafos, sob o Departamento
de Informática Aplicada (DIMAp) da Universidade Federal do Rio Grande
do Norte (UFRN).

Este relatório visa discutir a teoria relacionada a implementação,
isto é, revisar a teoria de grafos e discorrer sobre a estrutura e
construção da API. O relatório também pretende documentar
os testes de performance  nos algoritmos sob as representações de
grafos descritas na definição do projeto, com o objetivo de comparar
as estruturas e evidenciar empíricamente suas principais diferenças.

%%%%%%%%%%%%%%%%%%%%%%%%%%%%%%%%%%%%%%%%%%%%%%%%%%%%%%%%%%%%%%%%%%%%%%%%%%%%%%%%%%%
\section{Teoria dos grafos}
\label{sec:defs}

As definições utilizadas no projeto foram em grande parte retiradas
de \cite{diestel2025graph}, mas com algumas modificações.

\begin{mydef}[Grafo]
  Um \emph{grafo} é uma estrutura $G := \langle V, A \rangle$
  tal que $A \subseteq V^2$ e $V$ é um conjunto de um tipo qualquer.
  Os elementos de $V$ são denominados vértices (em inglês
  \textit{nodes}) e os elementos de $A$ são denominados de arestas
  (em inglês \textit{edges}). O jeito tradicional de visualizar um
  grafo é como uma figura composta de bolas e setas:
  \begin{figure}[h]
    \centering
    \begin{tikzpicture}
      \GraphInit[vstyle=normal]
      \tikzset{EdgeStyle/.style={->}}
      \Vertices{circle}{1,2,3,4}
      \Edges(1,2,3)
      \Edge(1)(4)
    \end{tikzpicture}
    \caption{Um grafo com $V := \{1,2,3,4\}$ e $A :=
    \{(1,2),(1,4),(2,3)\}.$}
    \label{fig:graph1}
  \end{figure}
\end{mydef}

\begin{mydef}[Ordem e Tamanho]
  O número de vértices de um grafo $G$ é chamado de \emph{ordem} e é
  denotado por $|G|$ -- o número de arestas é chamado de
  \emph{tamanho} e é denotado por $||G||$. Por exemplo, na
  figura~\ref{fig:graph1}, $|G| = 4$ e $||G|| = 3$.
\end{mydef}

\begin{mydef}[Adjacência]
  Dizemos que um vértice $v$ é adjacente de um vértice $u$ se somente
  se $(u,v) \in A$. Visualmente, enxergamos isso como:
  \begin{figure}[h]
    \centering
    \begin{tikzpicture}
      \GraphInit[vstyle=normal]
      \tikzset{EdgeStyle/.style={->}}
      \Vertices{circle}{v,u}
      \Edge(u)(v)
    \end{tikzpicture}
    \caption{Um grafo com $V := \{u,v\}$ e $A :=
    \{(u,v)\}.$}
  \end{figure}
\end{mydef}

% TODO:
% Definir:
% - Ordem e tamanho do um grafo
% - Vértices adjacentes
% - Conjunto de vizinhos
% - Grau de um vértice
% - Grafo não direcionado
% - Grafo subjacente

%%%%%%%%%%%%%%%%%%%%%%%%%%%%%%%%%%%%%%%%%%%%%%%%%%%%%%%%%%%%%%%%%%%%%%%%%%%%%%%%%%%
\section{Problem statement}
\label{sec:intro_prob_art}
This section describes the investigated problem in detail. You can
also have a separate chapter on ``Problem articulation.''  For some
projects, you may have a section like ``Research question(s)'' or
``Research Hypothesis'' instead of a section on ``Problem statement.'

%%%%%%%%%%%%%%%%%%%%%%%%%%%%%%%%%%%%%%%%%%%%%%%%%%%%%%%%%%%%%%%%%%%%%%%%%%%%%%%%%%%
\section{Aims and objectives}
\label{sec:intro_aims_obj}
Describe the ``aims and objectives'' of your project.

\textbf{Aims:} The aims tell a reader what you want/hope to achieve
at the end of the project. The  aims define your intent/purpose in
general terms.

\textbf{Objectives:} The objectives are a set of tasks you would
perform in order to achieve the defined aims. The objective
statements have to be specific and measurable through the results and
outcome of the project.

%%%%%%%%%%%%%%%%%%%%%%%%%%%%%%%%%%%%%%%%%%%%%%%%%%%%%%%%%%%%%%%%%%%%%%%%%%%%%%%%%%%
\section{Solution approach}
\label{sec:intro_sol} % label of Org section
Briefly describe the solution approach and the methodology applied in
solving the set aims and objectives.

Depending on the project, you may like to alter the ``heading'' of
this section. Check with you supervisor. Also, check what subsection
or any other section that can be added in or removed from this template.

\subsection{A subsection 1}
\label{sec:intro_some_sub1}
You may or may not need subsections here. Depending on your project's
needs, add two or more subsection(s). A section takes at least two
subsections.

\subsection{A subsection 2}
\label{sec:intro_some_sub2}
Depending on your project's needs, add more section(s) and subsection(s).

\subsubsection{A subsection 1 of a subsection}
\label{sec:intro_some_subsub1}
The command \textbackslash subsubsection\{\} creates a paragraph
heading in \LaTeX.

\subsubsection{A subsection 2 of a subsection}
\label{sec:intro_some_subsub2}
Write your text here...

%%%%%%%%%%%%%%%%%%%%%%%%%%%%%%%%%%%%%%%%%%%%%%%%%%%%%%%%%%%%%%%%%%%%%%%%%%%%%%%%%%%
\section{Summary of contributions and achievements} %  use this section
\label{sec:intro_sum_results} % label of summary of results
Describe clearly what you have done/created/achieved and what the
major results and their implications are.

%%%%%%%%%%%%%%%%%%%%%%%%%%%%%%%%%%%%%%%%%%%%%%%%%%%%%%%%%%%%%%%%%%%%%%%%%%%%%%%%%%%
\section{Organization of the report} %  use this section
\label{sec:intro_org} % label of Org section
Describe the outline of the rest of the report here. Let the reader
know what to expect ahead in the report. Describe how you have
organized your report.

\textbf{Example: how to refer a chapter, section, subsection}. This
report is organised into seven chapters. Chapter~\ref{ch:lit_rev}
details the literature review of this project. In
Section~\ref{ch:method}...  % and so on.

\textbf{Note:}  Take care of the word like ``Chapter,'' ``Section,''
``Figure'' etc. before the \LaTeX~command \textbackslash ref\{\}.
Otherwise, a  sentence will be confusing. For example, In
\ref{ch:lit_rev} literature review is described. In this sentence,
the word ``Chapter'' is missing. Therefore, a reader would not know
whether 2 is for a Chapter or a Section or a Figure.  For more
information on \textbf{automated tools} to assist in this work, see
\Cref{subsec:reftools}.
